\chapter{Conclusions and Future Work}
\label{conclusion}

\section{Conclusions}
\subsection{Parallel Computing 2D Voronoi Diagrams Using Sweepcircles}

This chapter presents the untransformed sweepcircle
algorithm for 2D Voronoi diagram. Our algorithm has the optimal
$O(n\log n)$ time complexity and $O(n)$ space complexity. The
classical sweep line algorithm is the degenerate form of our algorithm when the circle center is at infinity. Our
algorithm is flexible in that it allows multiple circles at
arbitrary locations to sweep the domain simultaneously, which
naturally leads to a parallel implementation. It is easy to
implement, without complicated numerical calculation. We demonstrate
the efficacy of our parallel sweep circle algorithm using GPU.

\subsection{Centroidal Voronoi Tessellation on Arbitrary Triangle Meshes}

Although Voronoi diagrams in Euclidean space have been well studied
 and understood, very little progress has been reported on computing
 Voronoi diagrams on curved surfaces. In this report, we developed
 an intrinsic algorithm to compute the geodesic Voronoi diagram~(GVD)
 on a surface, and used GVD  to iteratively
 compute a geodesic centroid Voronoi diagram on a surface.

\section{Future Work}

Computing intrinsic surface Voronoi Diagram
and Centroidal Voronoi Tessellation requires efficient
method to obtain geodesic distance. We developed an efficient
and parallel method for computing
discrete geodesic distance and plan to use it in calculating surface
Voronoi, see~\ref{subsec:svg} for reference.
Also, we can compute a Voronoi Diagram in $R^3$, and calculate
the intersection of a 3D Voronoi Diagram and a surface. The result
 is also a surface CVT. Also using
Euclidean distance in 3D is not appropriate because it ignores
the boundary condition of the surface.
Thus, we need an efficient way to calculate intrinsic distance
 in $R^3$, see~\ref{subsec:DIG} for reference.

\subsection{Fast and Parallel Method for Computing Discrete Geodesic Distance}
\label{subsec:svg}

To compute an intrinsic CVT on surfaces,
it is necessary to compute geodesic distances. However,
the existing methods for calculating geodesic distance
 are not fast enough.
As a fundamental problem in geometric modeling, computing geodesics
distance on triangle meshes has been widely studied.
 Many elegant algorithms have been presented to compute geodesic distance, e.g.,the CH
method~\cite{Chen_Han:1990}, the MMP
method~\cite{Mitchell_Etc:1987} and the ICH method~\cite{Xin_Wang:2009}, the fast marching
method~\cite{Kimmel_Sethian:1998} and the approximate MMP
algorithm~\cite{Surazhsky_Etc:2005}. While the state-of-the-art
approaches~\cite{Surazhsky_Etc:2005}~\cite{Xin_Wang:2009} work quite
well for models of moderate size, they are not practical for
large-scale models or time-critical applications due to their high
computational cost.

In order to compute geodesic distance field in real time,
we need a faster algorithm and a more efficient method. The past decade
has witnessed an increasing trend for the
traditionally CPU-handled computation to be performed using
 graphics processing unit
(GPU), which uses large numbers of graphics chips to parallelize the
computation. However, it is technically challenging to develop
parallel algorithms for discrete geodesics due to the lack of
parallel structure, i.e., the geodesic distance is propagated from
the source to all destination in a sequential order. So far, the
only parallel geodesic algorithm is that created by Weber et
al.~\cite{weber2008parallel}, who developed a
raster scan-based version of the fast marching algorithm. Although
being highly efficient, their method computes only the first-order
approximation of geodesic distance and also requires parameterizing the
surface into a regular domain, which is usually difficult for
surfaces with complicated geometry and/or topology.

We present a new data structure, called the \textit{Saddle
Vertex Graph} or SVG, which can be used to compute all types of
discrete geodesics. The SVG is an undirected graph that has the same
vertex set as the input triangle mesh. Each edge of the SVG corresponds
to a geodesic path that does not pass through any saddle vertices.
We call such a geodesic path \textit{direct}. The weight of the edge
is the length of the corresponding direct geodesic path. The
distinctive feature of the SVG is that it elegantly links the geodesic
problem on meshes with the shortest path problem on graphs, i.e.,
any geodesic path in the input mesh is the shortest path in the SVG.
As a result, we can apply the classical Dijkstra's shortest path
algorithm to compute all three types of discrete geodesics in a
unified framework, and avoids complex geometry calculations. Thus we
can use this method to compute intrinsic CVT in real time.

\subsection{Computing Discrete Geodesics Distance in $R^3$ for a Surface CVT}
\label{subsec:DIG}

We can compute a Voronoi Diagram in $R^3$, and calculate
the intersection of a 3D Voronoi Diagram and surface~\cite{Liu:2009:CVT}.
 The result is surface CVT.
 Using Euclidean distance in 3D is not quite suitable since it does not
  take the boundary condition into
account. Thus, we need an efficient  way to calculate intrinsic geodesic
distance in $R^3$, we plan to calculate the geodesic
 distance in tetrahedral meshes as a robust distance measurement.
  The distance field in tetrahedral meshes
 is very attractive in computer graphics and related fields.
It is used for multi-body dynamics~\cite{guendelman2003nonconvex},
 deformable objects~\cite{fisher2001deformed}, 
 mesh generation~\cite{molino2003crystalline},
 motion planning~\cite{hoff1999fast}, 
 and sculpting~\cite{baerentzen2002manipulation}.
 We will use it to calculate a 3D Voronoi Diagram, and
 use the 3D Voronoi Diagram to intersect with a surface 
 to get a surface CVT.

The conventional fast marching method is tessellation dependent
and very sensitive to the topological noise.
To tackle these challenges, we present an intrinsic method for computing
geodesics on tetrahedral meshes with noises.

We prompt this method by the following
intuition: Let $\Omega\subset\mathbb{R}^{3}$ be a planar domain
with noises. Given two points $q, p\in \Omega$, we can
calculate the Euclidean distance
$d(q,p)=\sqrt{(p_x-q_x)^2+(p_y-q_y)^2+(p_z-q_z)^2}$ 
easily, but the geodesic path $pq$ may
be hindered by the gaps. Put differently, if $\Omega$ is
planar, the shortest path inside $\Omega$ can pass \textit{through}
the holes. We
can change the classical wavefront algorithm based on this observation.
Fast marching method keeps a wave increasing
from the origin point to the target point. 
one can flatten the \textit{spatial}
boundary surfaces to the Euclidean space, when the wavefront gets 
the boundary surfaces.
Then we can  let the wavefront pass the boundaries by using the
Euclidean distance in the tetrahedron mesh. The wavefront keeps
increasing up the time when it gets all destinations.

